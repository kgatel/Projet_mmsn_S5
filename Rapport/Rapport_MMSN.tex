\documentclass[12,french]{report}
\usepackage{geometry}
\geometry{vmargin=3cm, hmargin=3cm}
\usepackage[T1]{fontenc}
\usepackage[utf8]{inputenc}
\usepackage[french]{babel}
\usepackage{graphicx}
\usepackage{amsmath}
\usepackage{amssymb}
\usepackage{sectsty}
\usepackage{authblk}
\usepackage{algpseudocode}
\usepackage{algorithm}
\usepackage{xspace}
\usepackage{mathtools}
\usepackage{mathrsfs}
\usepackage{enumitem}
\usepackage{titlesec}
\usepackage{hyperref}
\usepackage{xcolor}
\usepackage{caption}
\usepackage{float}
\usepackage{tabto}

\usepackage{listings}
\usepackage{cleveref}

\renewcommand{\lstlistingname}{Code}
%\renewcommand{\figurename}{Fig.}

\lstdefinestyle{chstyle}{%
backgroundcolor=\color{gray!12},
basicstyle=\ttfamily\small,
showstringspaces=false,
numbers=left}

%\AddThinSpaceBeforeFootnotes
%\FrenchFootnotes

\titleformat{\chapter}[hang]{\bf\Huge}{\thechapter.}{2pc}{}
\titlespacing*{\chapter}{10pt}{0pt}{40pt}[0pt]
\newcommand{\HRule}{\rule{\linewidth}{0.5mm}}

\providecommand{\keywords}[1]{\textbf{\textit{Keywords:}} #1}
\bibliographystyle{apalike}

\usepackage{hyperref}

\begin{document}
\hypersetup{pdfborder=0 0 0}

\begin{titlepage}

\begin{center}
	\vspace*{\stretch{1}}
	\textsc{{\LARGE Institut national des sciences appliquées de Rouen} \\ 			\vspace{6mm} {\Large INSA de Rouen}} \\
	\vspace{5mm}
	\includegraphics[width=0.4\textwidth]{./Images/insa}\\[1.0 cm]

	\textsc{\Large Projet MMSN GM3 - Vague 3 - Sujet 4}\\[0.6cm]

	% Title
	\HRule \\[0.5cm]
	{ \Huge \bfseries Résolution de système linéaire par la méthode du gradient conjugué}\\[0.2cm]
	\HRule \\[0.75cm]

	\includegraphics[width=0.6\textwidth]{./Images/Gradient_conjugué}\\[0.5 cm]

	% Author and supervisor
	\begin{minipage}{0.4\textwidth}
		\begin{flushleft} \large
			\emph{Auteurs:}\\
			Thibaut \textsc{André-Gallis} \\
			{\small\href{mailto:thibaut.andregallis@insa-rouen.fr}{thibaut.andregallis@insa-rouen.fr}} \\
			Kévin \textsc{Gatel} \\
			{\small\href{mailto:kevin.gatel@insa-rouen.fr}{kevin.gatel@insa-				rouen.fr}}
		\end{flushleft}
	\end{minipage}
	\begin{minipage}{0.4\textwidth}
		\begin{flushright} \large
			\emph{Enseignant:} \\
			Bernard \textsc{Gleyse} \\
			{\small\href{mailto:bernard.gleyse@insa-rouen.fr}								{bernard.gleyse@insa-rouen.fr}}
		\end{flushright}
	\end{minipage}
	\vspace*{\stretch{1}}

	\vfill
	{\large 4 Janvier 2021}
\end{center}
\end{titlepage}

\tableofcontents

%\listoffigures

\renewcommand{\chaptername}{}
\chapter*{Introduction}
%\label{chapter:Introduction}
\addcontentsline{toc}{chapter}{Introduction}

m1 m2

\chapter{Présentation du problème}

\section{Principe}

Expliquer le principe du problème

\section{Résolution mathématique}

Expliquer la résolution mathématique du problème (théorème sans démonstrations)

\chapter{Résolution numérique}

\section{Méthode}

Expliquer la méthode numérique utilisée (fortran algo etc)

\section{Résultats}

Convergence des $x_{n}$, convergence des résidus, p.s. des résidus qui forment bien une base, inégalité du conditionnement...

\chapter*{Conclusion}
\addcontentsline{toc}{chapter}{Conclusion}

Dans la conclusion, vous devez commenter les résultats numériques par rapport á ce que l’on pouvait espérer au vu des résultats théoriques.

\chapter*{Annexes}
\addcontentsline{toc}{chapter}{Annexe}

les noms de fichiers ( source, données,
résultats.

\chapter*{Bibliographie}
\addcontentsline{toc}{chapter}{Bibliographie}

[1] André Draux \textit{Analyse numérique}, poly, chapitre 2 \textit{Les méthodes de descente}.

\end{document}
